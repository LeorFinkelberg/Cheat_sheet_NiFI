\documentclass[%
	11pt,
	a4paper,
	utf8,
	%twocolumn
		]{article}	

\usepackage{style_packages/podvoyskiy_article_extended}


\begin{document}
\title{Сборник заметок по Apache NiFi}

\author{}

\date{}
\maketitle

\thispagestyle{fancy}

\tableofcontents

\section{Установка на Windows}

Для того чтобы установить NiFi следует скачать архив бинарников со страницы проекта \url{https://nifi.apache.org/download/} (например, \verb*|nifi-1.25.0-bin.zip|). Затем нужно распоковать архив в целевую директорию, например, в \directory{C: > Users > user > nifi}. 

После чего нужно запустить файл \verb*|run-nifi.bat|, расположенный в поддиректории \verb*|bin|.

В конфигурационном файле \directory{conf > nifi.properties} будет указан хост и порт
\begin{lstlisting}[
style = bash,
numbers = none
]
..
nifi.web.https.host=127.0.0.1
nifi.web.https.port=8443
...
\end{lstlisting}

А логин и пароль будут сгенерированы в файле \directory{logs > nifi-app.log}
\begin{lstlisting}[
style = bash,
numbers = none
]
...
Generated Username [e4ef4cdf-5615-4748-8280-d57de0558375]
Generated Password [X9WaNiKWzLhhpn5s4i/qzLQLwzRwUcKs]
...
\end{lstlisting}

\section{Основные термины и определения}

Полезные материалы

\noindent\url{https://nifi.apache.org/documentation/v1/} Официальная документация

\noindent\url{https://www.youtube.com/watch?v=yKXD1biS6V8&list=PL4MpKy3QjNp_rOEEibc4Ro8UK4g8vLX6_} Apache NiFi за 3 часа.

\noindent\url{https://www.youtube.com/watch?v=NC3IlbjrKeA&list=PLpe6CiYAs7vTS4LG7kkvuPZEzEobvw_lR} Apache NiFi

FlowFile -- основной объект Apache NiFI, содержащий атрибуты данных (метаданные) и контент.

FlowFile Attribute -- атрбуты FlowFile; храняться в оперативной памяти

FlowFile Content -- контент FlowFile; хранится на диске

Processor -- одно действие / опреация с данными FlowFile в Apache NiFi (<<прямоугольничек>> на схеме NiFi)

Process Group -- объединение Processors и других Process Groups

Process -- объединение Processors и Process Groups с целью реализации целевого алгоритма

\section{Архитектура Apache NiFi}

Apache NiFi состоит из:
\begin{itemize}
	\item веб-сервера,
	
	\item контроллера потоков и процессоров (встроенных или пользовательских),
	
	\item трех репоизториев: FlowFile Repository (хранит текущее состояние всех FlowFile'ов), Content Repository (конетнт всех FlowFile'ов) и Provenance Repository (хранит метрики, статистики и пр.).
\end{itemize}

Apache NiFi гарантирует строго однократную доставку. Если, например, выключат свет, то NiFi данные не потеряет, так как есть лог-файл. NiFi можно разворачивать в кластере для более быстрой параллельной загрузки и обработки данных.

Можно настроить процессор, чтобы он выполнялся либо на всех узлах, либо только на одном главном. Можно повесить ограничения на очередь. Скажем какая-то отдельно взятая очередь должна содержать не более 100 FlowFile'ов весом не более 10 Мб. Если условие нарушается, то NiFi приостановит процессор и таким образом высвободит ресурсы.

\section{Expression Language}

\begin{lstlisting}[
style = bash,
numbers = none
]
${abc} = значение атрибута
${abc:toUpper()}
$${abc} = "${abc}"  - экранирование
\end{lstlisting}

Ctrl +Space -- подсказки в UI по EL. Все функции можно посмотреть здесь \url{https://nifi.apache.org/docs/nifi-docs/html/expression-language-guide.html}

Типы данных:
\begin{itemize}
	\item String,
	
	\item Number, 
	
	\item Decimal,
	
	\item Date,
	
	\item Boolean
\end{itemize}

После вычислений значения атрибутов приводятся к String. Есть фунцкии приведения типов \verb*|$fileSize:toNumber()|, \verb*|${time:toDate()}|.

\section{Процессоры}

Процессоры в NiFi можно запускать по CRON-времени или через заданный промежуток времени. Есть удобный сервис CronMaker \url{http://www.cronmaker.com/;jsessionid=node09n1lwc8wiijupgvam6sbahzl1770037.node0?0}, который преобразует схему запуска в CRON-формат.

\subsection{JoltTransformJSON}

Очень мощный процессор. По кнопке <<\verb*|Advanced|>> открывается форма, в которой можно передать входной JSON, в поле <<Jolt Specification>> описать схему преобразования (пишется на специальном языке \url{https://jolt-demo.appspot.com/#inception}) и посмотреть как работает преобразование.


\section{Контроллер-сервис}

\subsection{DBCPConnectionPool}

Контроллер-сервису \verb*|DBCPConnectionPool| в поле <<\verb|Database Connection URL|>> нужно передать URL целевой СУБД, например, \verb*|jdbc:mysql://50.224.35.38:6974/usgs| (\verb*|usgs| -- это имя базы данных).

А в поле <<\verb|Database Driver Class Names|>> указать класс драйвера, например,

\noindent\verb*|com.mysql.cj.jdbc.Driver|

\noindent и путь до директории, в которой лежит драйвер (поле <<\verb|Database Driver Location(s)|>>), например,

\noindent\verb|/opt/nifi/nifi-current/custom-nar|.

Еще может потребоваться указать имя пользователя базы данных в поле <<\verb|Database User|>> и пароль.

Драйвер позволяет выполнять SQL-запросы к базе данных и зависит от ее типа, поэтому его нужно предварительно загрузить с официального сайта разработчиков (для MySQL \url{https://www.mysql.com/products/connector/}, а для PostreSQL \url{https://jdbc.postgresql.org/})




% Источники в "Газовой промышленности" нумеруются по мере упоминания 
\begin{thebibliography}{99}\addcontentsline{toc}{section}{Список литературы}	
	\bibitem{sobel:2011}{ \emph{Собель М}. Linux. Администрирование и системное программирование. 2-е изд. -- СПб.: Питер, 2011. -- 880 с. }
\end{thebibliography}

%\listoffigures\addcontentsline{toc}{section}{Список иллюстраций}

\lstlistoflistings\addcontentsline{toc}{section}{Список листингов}

\end{document}
