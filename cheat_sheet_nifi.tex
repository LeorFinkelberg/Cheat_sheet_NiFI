\documentclass[%
	11pt,
	a4paper,
	utf8,
	%twocolumn
		]{article}	

\usepackage{style_packages/podvoyskiy_article_extended}


\begin{document}
\title{Сборник заметок по Apache NiFi}

\author{}

\date{}
\maketitle

\thispagestyle{fancy}

\tableofcontents

\section{Установка}

Для того чтобы установить NiFi следует скачать архив бинарников со страницы проекта \url{https://nifi.apache.org/download/} (например, \verb*|nifi-1.25.0-bin.zip|). Затем нужно распоковать архив в целевую директорию, например, в \directory{C: > Users > user > nifi}. 

После чего нужно запустить файл \verb*|run-nifi.bat|, расположенный в поддиректории \verb*|bin|.

В конфигурационном файле \directory{conf > nifi.properties} будет указан хост и порт
\begin{lstlisting}[
style = bash,
numbers = none
]
..
nifi.web.https.host=127.0.0.1
nifi.web.https.port=8443
...
\end{lstlisting}

А логин и пароль будут сгенерированы в файле \directory{logs > nifi-app.log}
\begin{lstlisting}[
style = bash,
numbers = none
]
...
Generated Username [e4ef4cdf-5615-4748-8280-d57de0558375]
Generated Password [X9WaNiKWzLhhpn5s4i/qzLQLwzRwUcKs]
...
\end{lstlisting}






% Источники в "Газовой промышленности" нумеруются по мере упоминания 
\begin{thebibliography}{99}\addcontentsline{toc}{section}{Список литературы}	
	\bibitem{sobel:2011}{ \emph{Собель М}. Linux. Администрирование и системное программирование. 2-е изд. -- СПб.: Питер, 2011. -- 880 с. }
\end{thebibliography}

%\listoffigures\addcontentsline{toc}{section}{Список иллюстраций}

\lstlistoflistings\addcontentsline{toc}{section}{Список листингов}

\end{document}
